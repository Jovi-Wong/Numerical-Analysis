\documentclass[twoside,a4paper,12pt]{article}
\usepackage{geometry}
\geometry{margin=1.5cm, vmargin={0pt,1cm}}
\setlength{\topmargin}{-1cm}
\setlength{\paperheight}{29.7cm}
\setlength{\textheight}{25.3cm}

% useful packages.
\usepackage{amsfonts}
\usepackage{amsmath}
\usepackage{amssymb}
\usepackage{amsthm}
\usepackage{enumerate}
\usepackage{graphicx}
\usepackage{multicol}
\usepackage{fancyhdr}
\usepackage{layout}
\usepackage{amsmath} 

% some common command
\newcommand{\dif}{\mathrm{d}}
\newcommand{\avg}[1]{\left\langle #1 \right\rangle}
\newcommand{\difFrac}[2]{\frac{\dif #1}{\dif #2}}
\newcommand{\pdfFrac}[2]{\frac{\partial #1}{\partial #2}}
\newcommand{\OFL}{\mathrm{OFL}}
\newcommand{\UFL}{\mathrm{UFL}}
\newcommand{\fl}{\mathrm{fl}}
\newcommand{\op}{\odot}
\newcommand{\Eabs}{E_{\mathrm{abs}}}
\newcommand{\Erel}{E_{\mathrm{rel}}}
\begin{document}

\pagestyle{fancy}
\fancyhead{}
\lhead{Jovi Wong(3180104829)}
\chead{Numerical Analysis homework \#2}
\rhead{2020/3/28}

\section*{I.Linear Interpolation of $f(x)=\frac{1}{x}$ at $x_0=1$ and $x_1=2$}
\subsection*{a.Determine $\xi(x)$ explicitly} 
Using Lagrange Formula, we get
\[
p_{1}(x)=f_0\frac{x-x_0}{x_0-x_1}+f_1\frac{x-x_1}{x_1-x_0}=\frac{3-x}{2}
\]
Because $f''(x)=\frac{2}{x^3}$, it is obvious that $f''(\xi(x))=\frac{2}{\xi^3(x)}$, then 
\[
f(x)-p_1(x)=\frac{1}{x}-\frac{3-x}{2}=\frac{(x-1)(x-2)}{2x}=\frac{(x-1)(x-2)}{\xi^3(x)}
\]
Hence, we know that $\xi(x)=\sqrt[3]{2x}$.
\subsection*{b.Find max$\xi(x)$, min$\xi(x)$ and max$f''(\xi(x))$} 
For $x\in[1,2]$, we can conclude
\[
max\xi(x)=\xi(2)=\sqrt[3]{4}
\]
\[
min\xi(x)=\xi(1)=\sqrt[3]{2}
\]
From above, notice $f''(\xi(x))=\frac{2}{\xi^3(x)}$, so
\[
maxf''(\xi(x))=max\frac{1}{2x}=\frac{1}{2}
\]

\section*{II.Find $p\in\mathbb{P}_{2n}^{+}$ Satisfies Distinct Points on $\mathbb{R}$}
We can modify the Lagrange Formula into
\[
p(x)=\sum_{k=0}^{n}f_{k}l_{k}
\]
where the 
\[
l_k=\prod_{i=0;i \neq k}^{n}\frac{(x-x_i)^2}{(x_k-x_i)^2}
\]
and the degree of $p \leq {2n}$, which implies $p(x) \in \mathbb{P}$. It is obvious that $l_k \geq 0$, since $(x-x_i)^2\geq 0$ and $(x_k-x_i)^2 \geq 0$. Additionally, $f_k \geq 0$ given by the condition, therefore, $f_kl_k\geq 0$ so that $p(x) \geq 0$ has proved. 

\section*{III.Consider $f(x)=e^{x}$}
\subsection*{a.Prove Induction}
If $n=0$, namely, $f[t]=f(t)=e^t=\frac{(e-1)^n}{n!}e^t$, it is true obviously.\\
If this conclusion is true when $n=s-1$, then we can know $\forall t\in \mathbb{R}$
\[
f[t,t+1,、\cdots,t+s-1]=\frac{(e-1)^{s-1}}{(s-1)!}e^{t}
\]
Next, we can subsititute $t$ with $t+1$, so we can get
\[
f[t+1,t+2,、\cdots,t+s]=\frac{(e-1)^{s-1}}{(s-1)!}e^{t+1}
\]
Using therorem 3.14, we can draw the conlcusion
\[
f[t,t+1,、\cdots,t+s]=\frac{\frac{(e-1)^{s-1}}{(s-1)!}e^{t+1}-\frac{(e-1)^{s-1}}{(s-1)!}e^{t}}{s}=\frac{(e-1)^s}{s!}e^t
\]
It implies when $n=s$ the formula is also true. Hence proved.
\subsection*{b.Determine $\xi$ From the Above Two Equation}
From above, take $t=0$ into the equation,
\[
f[0,1,\cdots,n]=\frac{(e-1)^n}{n!}
\]
When $x \in (0,n)$, $\frac{e^x}{n!} \in (\frac{1}{n!},\frac{e^n}{n!})$. Since $e^x$ is continous function, we know $\exists \xi \in (0,n)$ such that 
\[
\frac{e^{\xi}}{n!}= \frac{(e-1)^n}{n!}
\]
by Intermediate Value Theorem. Therefore,
\[
x=n \ln{(e-1)} > \frac{n}{2}
\]
which means that $\xi$ always lays on the right of the midpoint $x=\frac{n}{2}$ .
\section*{IV.Cosider $f(0)=5$, $f(1)=3$, $f(3)=5$, $f(4)=12$}
\subsection*{a.Obtain $p_3(f;x)$ Using Newton Formula}
According to the formula,
\[
f[0]=5
\]
\[
f[0,1]=\frac{f_0}{x_0-x_1}+\frac{f_1}{x_1-x_0}=-2
\]
\[
f[0,1,3]=\frac{f_0}{(x_0-x_1)(x_0-x_2)}+\frac{f_1}{(x_1-x_0)(x_1-x_2)}+\frac{f_2}{(x_3-x_0)(x_3-x_2)}=\frac{5}{6}
\]
\[
f[0,1,3,4]=\frac{f_0}{(x_0-x_1)(x_0-x_2)(x_0-x_3)}+\frac{f_1}{(x_1-x_0)(x_1-x_2)(x_1-x_3)}
\]
\[
+\frac{f_2}{(x_2-x_0)(x_2-x_1)(x_2-x_3)}+\frac{f_3}{(x_3-x_0)(x_3-x_1)(x_3-x_2)}=\frac{1}{4}
\]
Easily, we can get
\[
p_3(x)=a_0+a_1(x-x_0)+a_2(x-x_0)(x-x_1)+a_3(x-x_0)(x-x_1)(x-x_2)
\]
\[
=\frac{x^3}{4}-\frac{9}{4}x+5
\]
\subsection*{b.Find an Appropriate Value For the location $x_{min}$ of the Minimum}
When $f'(x)=\frac{3}{4}x^2-\frac{9}{4}=0$, we know $x=\sqrt{3}$. It implies 
\[
x_{min}=\sqrt{3} ,minf(x)=5-\frac{3}{2}\sqrt{3}
\]
\section*{V.Consider $f(x)=x^{7}$}
\subsection*{a.Compute $f[0,1,1,1,2,2]$}
\begin{tabular}{c|cccccc}
x&0&1&1&1&2&2\\
\hline
$f(x)$&0&1&1&1 &128& 128\\

\end{tabular}
\\
\\
Using therorem 3.14, then we can get the chart recursively\\ 
\\
\begin{tabular}{c|rrrrrr}
0 & 0\\
1 & 1 & 1\\
1 & 1 &7 & 6\\
1&1&7&21&15\\
2&128&127&120&99&42&\\
2&128&448&321&201&102&30\\
\end{tabular}
\\
\\
which shows that $f[0,1,1,1,2,2] = 30$ .
\subsection*{b.Divided Difference of $f^{(5)}$ Evaluated at $\xi\in(0,2)$. Determine $\xi$}
According to corollary 3.17,
\[
f[0,1,1,1,2,x]=\frac{1}{5!}f^{(5)}(\xi(x))
\]
when $x=2$, namely, $\frac{1}{5!}\cdot2520\cdot\xi^2=30$, so $\xi=\frac{\sqrt{70}}{7}$ .
\section*{VI.$f$ is defined on [0,3] with conditions}
\subsection*{a.Estimate $f(2)$ Using Hermite Interpolation}
\begin{tabular}{c|rrrrr}
x & 0 & 1 & 1 & 3 & 3\\
\hline
$f(x)$& 1 & 2 & 2&0&0\\
\end{tabular}
\\
\\
\begin{tabular}{c|rrrrr}
0& 1\\
1& 2& 1\\
1& 2& -1& -2\\
3& 0& -1& 0& $\frac{2}{3}$\\
3& 0& 0& $\frac{1}{2}$ &$\frac{1}{4}$ &-$\frac{5}{36}$\\
\end{tabular}
\\
Therefore, $p(x)=1+x-2x(x-1)+\frac{2}{3}x(x-1)^2-\frac{5}{36}x(x-1)^2(x-3)$, in particular, $p(2)=\frac{11}{18}$ .
\subsection*{b.Estimate the Maximum Possible Error}
According to therorem 3.27, in particular, when $x=2$ ,
\[
\epsilon \leq \frac{M}{60}
\]
In general,
\[
f(x)-p(x)=\frac{f^{(5)}(\xi)}{5!}x(x-1)^{2}(x-3)^2=\frac{f^{(5)}(\xi)}{5!}(x^5-8x^4+22x^3-24x^2+9x)
\]
Take
\[
h(x)=x^5-8x^4+22x^3-24x^2+9x
\]
then,
\[
h'(x)=(x-1)(x-3)(5x^2-12x+3)
\]
After trivial calculation, the maximum of possible error  
\[
\epsilon=\max_{x\in(0,3)}{|f(x)-g(x)|}\leq\frac{M}{120}h(\frac{6+\sqrt{21}}{5})=\frac{204+14\sqrt{21}}{15625}M
\]
\section*{VII.Prove Two Equation about Forward and Backward Difference}
When $k=1$ , then
\[
\bigtriangleup f(x)=f(x+h)-f(x)=hf[x_0,x_1]
\]
satisfies the conclusion.
If the conclusion is true, when $k=s-1$, namely,
\[
\bigtriangleup^{s-1}f(x)=(s-1)!h^{s-1}f[x_0,\cdots,x_{s-1}]
\]
\[
\bigtriangleup^{s-1}f(x+h)=(s-1)!h^{s-1}f[x_1,\cdots,x_{s}]
\]
Then according to the definition,
\[
\bigtriangleup^{s}f(x)=\bigtriangleup^{s-1}f(x+h)-\bigtriangleup^{s-1}f(x)
\]
\[
=(s-1)!h^{s-1}f[x_1,\cdots,x_{s}]-(s-1)!h^{s-1}f[x_0,\cdots,x_{s-1}]
\]
\[
=(s-1)!h^{s-1}f[x_1,\cdots,x_{s}]-(s-1)!h^{s-1}f[x_0,\cdots,x_{s-1}]
\]
\[
=(s-1)!h^{s-1}(f[x_1,\cdots,x_{s}]-f[x_0,\cdots,x_{s-1}])
\]
According to therorem 3.14,
\[
f[x_1,\cdots,x_{s}]-f[x_0,\cdots,x_{s-1}]=khf[x_0,\cdots,x_k]
\]
So take it into the other formula,
\[
\bigtriangleup^{s}f(x) = s!h^{s}f[x_0,\cdots,x_{s}])
\]
The situation of backward difference is similar to the above, hence proved.

\section*{VIII.Prove equation and Expand It}
\subsection*{a.prove equation}
When $n=0$, $LHS=\frac{\partial}{\partial{x_0}}f[x_0]=f'(x_0)$ and $RHS = f[x_0,x_0]=f'(x_0)$, hence the equation holds.\\
When $n=k$, suppose $\frac{\partial}{\partial{x_0}}=f[x_0,x_0,\cdots,x_k]$ is true,\\
Then, when $n=k+1$,
\begin{gather}
LHS=\frac{\partial}{\partial{x_0}}\frac{f[x_1,x_2,\cdots,x_{k+1}]-f[x_0,x_1,\cdots,x_k]}{x_{k+1}-x_0} \\
=\frac{f[x_1,x_2,\cdots,x_{k+1}]}{(x_{k+1}-x_0)^2}-\frac{f[x_0,x_0,\cdots,x_k](x_{k+1}-x_0)+f[x_1,x_2,\cdots,x_k]}{(x_{k+1}-x_0)^2}\\
=\frac{f[x_1,x_2,\cdots,x_{k+1}]-f[x_0,x_1,\cdots,x_k]}{(x_{k+1}-x_0)^2}-\frac{f[x_0,x_0,\cdots,x_k]}{x_{k+1}-x_0}\\
=\frac{f[x_0,x_1,\cdots,x_{k+1}]-f[x_0,x_0,\cdots,x_k]}{x_{k+1}-x_0}\\
=f[x_0,x_0,x_1,\cdots,x_{k+1}]\\
=RHS
\end{gather}
Hence proved.
\subsubsection*{b.expand it}
The proof of the following equation is simialr to the above content that just exchange $x_0$ with $x_i$ , then we can get  
\begin{gather}
\frac{\partial}{\partial{x_i}}f[x_0,x_1,\cdots,x_i,\cdots,x_{n}]=f[x_i,x_0,x_1,\cdots,\cdots,x_i,\cdots,x_n]
\end{gather}
\end{document}

%%% Local Variables: 
%%% mode: latex
%%% TeX-master: t
%%% End: 
